\documentclass[12pt,notitlepage]{amsart}
\usepackage{amssymb}
\usepackage{amsmath}
\usepackage{tikz}
\usetikzlibrary{arrows,positioning}
\newtheorem{theorem}{Theorem}
\newtheorem{lemma}{Lemma}
\newtheorem{claim}{Claim}
\newtheorem{deftemp}{Definition}
%\newenvironment{proof}{\par \noindent \textsc{Proof.}}{\hfill $\square$ \vskip.1in}
\newenvironment{definition}{\begin{deftemp} \normalfont}{\end{deftemp}}
%\def\orb{\text{Orb}}
%\def\fix{\text{fix}}
%\pagestyle{empty}
\usepackage{bbm}
\newtheorem{proposition}{Proposition}
\newtheorem{corollary}{Corollary}
%\theorembodyfont{\rmfamily}
%\newenvironment{proof}{\smallskip\noindent\textsc{Proof.}\hskip.1in}
%{\hfill$\square$\par\smallskip}
\DeclareMathOperator{\fix}{{\rm fix}}
\DeclareMathOperator{\Sym}{{\rm Sym}}
\DeclareMathOperator{\Orb}{{\rm Orb}}
\DeclareMathOperator{\dom}{{\rm dom}}
\def\seq#1{\langle #1 \rangle}
\hoffset-1.in \textwidth7.in \voffset -.75in \textheight 9.25in
\begin{document}
\begin{theorem}
In $\mathcal{N}(3)$ every almost disjoint family can be extended to a maximal almost disjoint family.
\end{theorem}

\begin{proof}
Assume that $\mathcal{A}$ is an almost disjoint family of subsets of $X$ in $\mathcal{N}(3)$ with (finite) support $E$.  We may assume without loss of generality that $\mathcal{A}$ is maximal among almost disjoint families in $\mathcal{N}(3)$ with support $E$.  
\par\noindent\textbf{Claim.}  $\mathcal{A}$ is a maximal almost disjoint family in $\mathcal{N}(2)$.
\par\noindent\textbf{Proof.}  Assume $\mathcal{A}$ is not maximal and that $B$ is a subset of $X$ which is in $\mathcal{N}(3)$ but not in $\mathcal{A}$ for which $\mathcal{A} \cup \{ B  \}$ is an almost disjoint family.  Assume that the (least) support of $B$ is $E'$.  Since $\mathcal{A}$ is maximal among almost disjoint families with support $E$, there is an atom $a_0 \in E' \setminus E$.

We first note that 
\begin{equation} \label{E:forallphi}
\forall \psi \in \fix_G(E), \forall A \in \mathcal{A}, \psi(B) \cap A \mbox{ is finite.}
\end{equation}
(If $\psi(B) \cap A$ is infinite for some $A \in \mathcal{A}$ then $\psi^{-1}(\psi(B) \cap A) = B \cap \psi^{-1}(A)$ is infinite.  This contradicts the almost disjointness of $\mathcal{A} \cup \{ B \}$ since $\psi^{-1}(A) \in \mathcal{A}$.)


\end{proof}

\end{document}
